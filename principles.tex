\chapter{原理 \label{chap:principles}}
\section{深層学習 \label{section:deep-learning}}
深層学習は機械学習の一分野であり、複数の隠れ層から成るニューラルネットワークを使用して高度なパターン認識や特徴抽出を行う手法である。この章では最も単純で典型的な深層学習のアーキテクチャを説明する.まず,ニューラルネットワークは, 膨大な量のデータを用いてニューラルネットワークをトレーニングし、重みやバイアスなどのパラメータを調整して入力データから目的の出力を生成するように学習します。この過程では、バックプロパゲーションと呼ばれる手法を用いて、出力の誤差を最小化するためにネットワーク内の重みを調整します。

深層学習の特徴は、大規模で非線形な問題に適用できることや、画像認識、音声認識、自然言語処理などの様々な領域で驚異的な性能を示すことです。また、GPUなどの高性能な計算リソースが利用可能になったことで、大規模なニューラルネットワークのトレーニングが実用的になりました。

深層学習はその柔軟性と高い表現力により、現代の多くの応用分野で革新的な成果をもたらしており、これからもその進化と応用範囲の拡大が期待されています。

\section{格子ボルツマン法 \label{section:lbm}}