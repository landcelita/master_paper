\chapter{結論}

短期風速予測において,本研究で提案した手法は他の手法と比較して一定度優れた予測精度を示すことができた.特に,台風における風速予測においてその物理的構造がもつ特性を活かせることを確認した.また,第\ref{chap:introduction}章で述べたように,風力発電時における突発的な強風を予測することは,風力発電の運用において重要な役割を果たす.本研究で提案した手法は大きく風速が変化する場合においても予測が優れていたことから,風力発電の運用において有用であると言えるだろう.

一方で,いくつかの課題点や改善点が見つかった.一点目に,本実験において入出力データの平均化をおこなってしまったことで詳細な地形的な影響を反映できなかった点が挙げられる.陸海の境界や山脈などの地形的な影響を反映するためには,より高解像度のデータを用いることが望ましい.大きなデータセットのまま用いるのではなく,一部の陸地領域を切り出して高解像度のデータを用いることで,計算量を小さくしたまま地形的な影響を反映することができるだろう.この場合,遠方の風速情報が影響しないようにタイムステップを短くするなどの工夫が必要になる.二点目に,提案モデルは並進のたびに外枠の格子が切り捨てられてしまうのでモデルの出力を再入力して更に未来を予測させるということができない点がある.これは局所的な短期風速予測の宿命でもあるが,地球上の全地域を長方形の形状に分割しそれぞれの地域において時間発展させることでモデルの入出力を更に再帰的にできるようになるだろう.最後に,提案モデルには時間発展とともに風速を強めるような機構が不足していることを述べた.この解決策として格子ボルツマン法でいうところの外力項\cite{Inamuro2020}を重みとともに導入し学習させるということが考えられる.
