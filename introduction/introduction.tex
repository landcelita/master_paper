\chapter{はじめに}

\section{研究背景}
本研究の背景には,ニューラルネットワークを用いた深層学習と格子ボルツマン法(Lattice Boltzmann Method, LBM)の二つの重要な要素がある.深層学習は機械学習の一分野であり,複数の隠れ層を持つニューラルネットワークを使って複雑なパターンを学習する手法である.近年盛んに研究されていて応用先の一つに気象予測があり\cite{Schultz2021},既存の気象予測システムに比べて計算時間が短い上,より正確な予測が可能なモデルも報告されている\cite{Google2023}.他方で,格子ボルツマン法は比較的新しい数値流体力学の手法であり,個々の粒子のふるまいを扱うのではなく,格子状に分割した空間内で各格子点上の離散化された速度分布関数を解くものである\cite{doi:10.1146/annurev.fluid.30.1.329}.この手法の長所として並列計算と相性がよいことが挙げられ,GPUやTPUのようなプロセッサ上で効率的に計算することができる.

\section{先行研究}
TBD

\section{研究目的}
本研究では,ニューラルネットワークを利用することで従来物理シミュレーションが苦手としてきた[[要出典]]地表付近での風速予測の精度向上を図る.また,LBMとニューラルネットワークの手法をただ組み合わせるだけではなく第\ref{chap:how-to-assemble}章に述べるように構造のレベルで融合させることで,これら手法の親和性の高さを確かめる.

\section{論文構成}
TBD 